\section{Der Controller}

Folgende Gleichung bildet das Herzstück der Arbeit, den eigentlichen Controller:\\

$u_i^\alpha=\sum\limits_{R_j\in N_i^1}\phi_\alpha(||q_j^i||_\sigma)n_{ij}^i+
\sum\limits_{R_j\in N_i^1}p_j^i$\\\\
Diese Gleichung, die Alpha-Control, ist nur ein Teil einer Summe mehrerer Controls. Da die anderen beiden
(Beta- und Gamma-Control) jedoch auf zentralisierte Methoden beruhen, werden diese für die Arbeit nicht
betrachtet. Die Alpha-Control gibt einfach ausgedrückt die Distanz und Ausrichtung zwischen Robotern an,
die eingehalten werden muss, um aus dem Schwarm ein Dreieck-Gitter mit gewünschter Größe zu machen, 
ähnlich wie ein Netz oder eine künstliche Kraft. Sie bestimmt also, wie 'straff' dieses Netz sein soll. 
Ist es zu straff, reißt die Verbindung zwischen Robotern eventuell ab. Ist es zu schwach, stoßen Roboter 
mit hoher Wahrscheinlichkeit ineinander. Im Optimum konvergieren beide Teiler dieser Gleichung gegen 
Null, wenn der Schwarm perfekt ausgerichtet ist. Der erste Teil der Gleichung ist dabei die tatsächliche
Kraft im Schwarm, der zweite Teil die Ausrichtung der Roboter im Schwarm. Ziel ist es nun, dass jeder
Roboter seine Werte durch Bewegung von oder zum Nachbarn so anpasst, dass die Alpha-Control optimal wird.
Für die Action-Function $\phi_\alpha(z)$ reicht es zu wissen, dass diese mittels Parameter eingestellt
werden kann, um die Stärke im Schwarm zu definieren, ähnlich wie die Stärke eines Elektromagneten.