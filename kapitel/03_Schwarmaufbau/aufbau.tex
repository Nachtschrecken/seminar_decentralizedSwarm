\section{Der Schwarm}

Wie bereits erwähnt sind Graphenalgorithmen für die Kommunikation innerhalb des Schwarms essentiell.
Wir befassen uns jetzt erst einmal mit der "Sicht" eines Roboters $R_i$ auf den Schwarm. Ein einzelner 
Roboter kennt nicht den gesamten Schwarm, sondern nur seine eigenen 2-Hop-Nachbarschaft. Logischerweise
erfasst $R_i$ seine 1-Hop-Nachbarn durch die Sensoren und den einfachen Austausch im Signalradius von 
einem Meter. Die 2-Hop-Nachbarn werden bei jedem anpingen eines Nachbarn von diesem Nachbarn $R_j$ (wir
nennen den 'Nachbarn' ab nun $R_j$) nun als Array seiner eigenen 1-Hop-Nachbarn mit übertragen.
So erhält $R_i$ seine 2-Hop-Nachbarschaft und aktualisiert diese ebenfalls ständig durch Abgleich mit
Nachbarn im Kommunikationsradius. Der Schwarm ist also ein Zusammenschluss von subjektiven 
2-Hop-Nachbarschaften eines jeden Roboters. Die Kommunikation im Schwarm erfolgt mittels verteilter
Breitensuche, welche wir gleich betrachten.