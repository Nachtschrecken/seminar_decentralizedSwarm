\subsection{Verteilte Breitensuche}

Die Verteilte Breitensuche von Li et al. und McLurkin ist ein Algorithmus, um einen Baum in einem 
Graphen von einer Wurzel aus aufzuspannen. Jeder Wurzelknoten $R_0$ setzt dabei seinen initialen Hop 
auf Null und produziert eine Nachricht mit initialem Zeitstempel Null, welche mit jedem Hop um eins 
erhöht wird. Die Empfänger $R_i$ setzen ihren Hop für die nächste Nachricht nun um eins höher als empfangen 
und schicken diese an alle Nachbarn $R_j$. Die Nachbarn speichern nur die Sender mit geringster Distanz und 
der Schwarm $G$ baut so einen Baum auf. Durch diese Struktur können nun Nachrichten im Schwarm versendet
werdern oder auch anhand der Anzahl der Hops die Entgernung von Robotern bestimmt werden.