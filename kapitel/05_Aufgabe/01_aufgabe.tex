\section{Aufgabe des Schwarms}

Die Aufgabe des Schwarms ist es nun die Alpha-Control im Optimum zu halten, aber auch 
Hindernissen auszuweichen und eine interne Verbindung beizubehalten. Um die Alpha-Control umzusetzen 
gibt es zwei Werte: Das Paarweise Potential für den ersten Teil der Alpha-Control und das Laplace 
Potential für den zweiten Teil. Das \textbf{Paarweise Potential} $\eta_i^q$ gibt an, zu welchem Grad die Kanten 
in einem Dreieck-Gitter dem gewünschten, definierten Abstand der jeweiligen Roboter $R_i$ entsprechen. 
Für diese Arbeit wurde $d_{desire}=0.74m$ gewählt, damit die Roboter in ihrem Kommunikationsradius 
von einem Meter bleiben. Das \textbf{Laplace Potential} $\eta_i^\theta$ gibt an, zu welchem Grad die Roboter 
eines Schwarms $G$ in dieselbe Richtung ausgerichtet sind. Im Idealfall konvergieren beide diese Werte 
gegen Null. Wenn Beide Potentiale also gleich Null sind,
haben die Roboter im Schwarm den perfekten Abstand zueinander und sind alle in dieselbe Richtung
ausgerichtet. Mit diesen beiden Potentialen kann man nun die Hauptaufgaben des Schwarms 
definieren. Diese lauten folgendermaßen:

\begin{enumerate}
    \item Nach dem Ausweichen sollte jeder Roboter $R_i$ in einem Hindernis-freien Raum sein
    \item Nach dem Ausweichen sollte der Schwarm $G$ seine Ausgangsposition einnehmen
    \item Während des Ausweichens sollte der Schwarm $G$ die interne Verbindung beibehalten
    \item Nach dem Ausweichen sollte der Schwarm $G$ nicht mehr dem Hindernis zugewandt sein
\end{enumerate}

Wichtig für die Erkennung von Hindernissen und der Kommunikation im Schwarm sind außerdem Lokale 
Gelenkpunkte und die verteilte Breitensuche.