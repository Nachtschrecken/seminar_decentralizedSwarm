\section{Einleitung}

Die Arbeit beschäftigt sich mit dem Erkennen und Ausweichen von Hindernissen eines autonomen 
Roboterschwarms. Diese Roboter könne ihre Umgebung nicht selber wahrnehmen, sondern tasten Hindernisse als
Netz ab. Ziel dieser Arbeit ist es, einen Controller in Form eines Automaten zu entwicklen, durch welchen
die Roboter Umgehungsstrategien anwenden und sich ohne Verbindungsabbrüche im Raum bewegen können. Der
Schwarm agiert dabei dezentral, das heißt es gibt weder eine zentrale Steuereinheit, noch GPS oder 
Kartensysteme mit denen sich der Schwarm orientiert. Dieses Vorgehen kann kostengünstig und effizient
umgesetzt werden, da auch einzelne Roboter ausfallen können, ohne, dass der gesamte Schwarm ausfällt. Die
Roboter kommunizieren untereinander in einer Baumstruktur, Graphenalgorithmen sind für diese Arbeit also
ein essentieller Bestandteil. Nachdem die Roboter und ihr Aufbau als Schwarm vorgestellt werden, wird das
Herzstück dieses Controllers behandelt, der Phasenübergangsautomat. Am Ende werden Simulationsergebnisse
ausgewertet, um die Funktion des Controllers zu garantieren.