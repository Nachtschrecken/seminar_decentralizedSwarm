\section{Schluss}

Der Autor dieser Arbeit hat die Funktion dieses Controllers durch Simulation geprüft und konnte feststellen,
dass der Schwarm eine interne Verbindung aufrecht erhält und alle vier Kriterien der Aufgabe erfüllt werden
konnten. Auch wenn die Dauer des Manövers und die Effizienz mit steigender Anzahl an Robotern abnimmt, konnte
eine Möglichkeit gefunden werden einen autonomen Schwarm so agieren zu lassen, dass er ohne zentralisierte
Methoden überleben kann. Anwendungsgebiete für dieses Vorgehen finden sich vor allem in der Erkundung von
Terrain, das in der manuellen Erkundung durch Menschen oder sensible zentralisierte Schwärme viel zu teuer 
wäre, so etwa in der Antarktis. Mit entsprechender Hardware ausgestattet könnte ein solcher Schwarm im Flug
in Bergungsmissionen während Überflutungen oder anderen Szenarien eingesetzt werden.